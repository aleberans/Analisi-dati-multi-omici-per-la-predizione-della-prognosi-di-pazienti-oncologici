\documentclass[12pt,italian]{report}
\usepackage{tesi}

%
%			INFORMAZIONI SULLA TESI
%			DA COMPILARE!
%

% CORSO DI LAUREA:
\def\myCDL{Corso di Laurea magistrale in Informatica}

% TITOLO TESI:
\def\myTitle{}

% AUTORE:
\def\myName{Alessandro Beranti}
\def\myMat{Matr. Nr. 977702}

% RELATORE E CORRELATORE:
\def\myRefereeA{Prof. Elena Casiraghi}
\def\myRefereeB{Prof. Dario Malchiodi}

% ANNO ACCADEMICO
\def\myYY{2021-2022}

% Il seguente comando introduce un elenco delle figure dopo l'indice (facoltativo)
%\figurespagetrue

% Il seguente comando introduce un elenco delle tabelle dopo l'indice (facoltativo)
%\tablespagetrue

%
%			PREAMBOLO
%			Inserire qui eventuali package da includere o definizioni di comandi personalizzati
%

% Package di formato
\usepackage[a4paper]{geometry}		% Formato del foglio
\usepackage[italian]{babel}			% Supporto per l'italiano
\usepackage[utf8]{inputenc}			% Supporto per UTF-8
\usepackage[a-1b]{pdfx}			% File conforme allo standard PDF-A (obbligatorio per la consegna)
\usepackage[pdfa]{hyperref}

% Package per la grafica
\usepackage{graphicx}				% Funzioni avanzate per le immagini
\usepackage{hologo}					% Bibtex logo with \hologo{BibTeX}
%\usepackage{epsfig}				% Permette immagini in EPS
\usepackage{xcolor}				% Gestione avanzata dei colori

% Package tipografici
\usepackage{amssymb,amsmath,amsthm} % Simboli matematici
\usepackage{listings}				% Scrittura di codice

% Package ipertesto
\usepackage{url}					% Visualizza e rendere interattii gli URL
\usepackage{hyperref}				% Rende interattivi i collegamenti interni
\usepackage{caption}
\usepackage{booktabs}
\usepackage{verbatim}
\usepackage{adjustbox}
\usepackage{makecell}
\usepackage{colortbl}
\lstset{language=Python} 
\setcounter{tocdepth}{4}
\setcounter{secnumdepth}{4}

\begin{document}
	
	% Creazione automatica del frontespizio
	\frontespizio
	\beforepreface
	
	% 
	%			PAGINA DI DEDICA E/O CITAZIONE
	%			facoltativa, questa è l'unica cosa che dovete formattare a mano, un po' come vi pare
	%
	
	{\raggedleft \large \sl to do\\
		
	}
	
	
	
	
	% 
	%			PREFAZIONE (facoltativa)
	%
	
	%\prefacesection{Prefazione}
	%Le prefazioni non sono molto comuni, tuttavia a volte capita che qualcuno voglia dire qualcosa che esuli dal lavoro in s\'e (come un meta-commento sull'elaborato), o voglia fornire informazioni riguardanti l'eventuale progetto entro cui la tesi si colloca (in questo caso \`e probabile che sia il relatore a scrivere questa parte).
	
	%
	%			RINGRAZIAMENTI (facoltativi)
	%
	
	\prefacesection{Ringraziamenti}
	to do
	
	%
	%			Creazione automatica dell'indice
	%
	
	\afterpreface
	
	% 
	%			CAPITOLO 1: Introduzione o Abstract
	% 
	
	\chapter{Machine Learning}
	
	\section{Apprendimento supervisionato}
	\subsection{Decision Tree Classifier}
	\subsection{Random Forest Classifier}
	
	\chapter{Dataset}
	\section{The Cancer Genome Atlas (TCGA)}
	
	\chapter{Feature selection}
	\section{Tecniche univariate}
	\subsection{Bassa variabilità}
	\subsection{Kruskal-Wallis}
	\subsection{Mann-Whitney}
	\section{Tecniche multivariate}
	\subsection{Minimum Redundancy Maximum Relevance: mrmr}
	\subsection{Boruta}
	\section{Dimensionalità intrinseca}
	\subsection{ID\_twoNN}
	\section{Maximal information-based nonparametric exploration (MINE)}
	\subsection{The maximal information coefficient (MIC)}
	
	\chapter{Feature extraction}
	\section{Uniform Manifold Approximation: umap}
	
	\chapter{Esperimenti}
	\section{Preprocessing}
	\subsection{Scalare i dati}
	
	\section{Model selection}
	\subsection{Tuning degli iperparametri}

	\section{Cross Validation}
	
	\section{Metrica di performance}
	\subsection{Dati sbilanciati}
	\subsection{Area sotto la curva precision-recall}
	
	\section{Analisi dei risultati}
	
	\section{Tecnologie usate}
	
	
	%
	%			BIBLIOGRAFIA
	%
	
	\bibliographystyle{unsrt}
	\bibliography{bibliografia}
	\addcontentsline{toc}{chapter}{Bibliografia}
	
\end{document}
