\documentclass[12pt]{article}
\usepackage[utf8]{inputenc}
\usepackage[italian]{babel}
\usepackage[T1]{fontenc}
\usepackage[bitstream-charter]{mathdesign}
\usepackage[left=1.2in, right=1.2in, top=1in, bottom=1in, includefoot, headheight=13.6pt]{geometry}
\usepackage[a-1b]{pdfx}

\usepackage{setspace}
\onehalfspacing 
\usepackage[pdfa]{hyperref}

\begin{document}
	
	\title{{\Large
			\textbf{\textmd{Analisi di dati multi-omici per la predizione\\ \vspace{0.5cm} della prognosi di pazienti oncologici}}}}
	
	\author{Alessandro Beranti - 977702}
	
	\date{Febbraio 2023}
	
	\maketitle
	
	\vspace{1cm}
	Durante il periodo di tesi ho lavorato all'interno di AnacletoLAB, laboratorio focalizzato sull'applicazione di metodi di Intelligenza Artificiale su Biologia Molecolare e Medicina facente parte del Dipartimento di Informatica dell'Università degli Studi di Milano.
	I geni BRCA1 e BRCA2 sono importanti per la salute dei tessuti della mammella e dell'ovaio. Questi geni sono normalmente responsabili della produzione delle proteine che proteggono dalle mutazioni del DNA e prevengono la crescita di cellule anormali. Tuttavia, quando i geni BRCA1 e BRCA2 sono mutati, le proteine che producono non funzionano correttamente e c'è un rischio maggiore di sviluppare tumori al seno e all'ovaio. Le mutazioni dei geni BRCA1 e BRCA2 sono anche associate a un aumento del rischio di tumori della prostata, del pancreas e dello stomaco.
	
	Il lavoro ha riguardato l'uso di tecniche di apprendimento supervisionato, in particolare usando come algoritmo \textit{random forest} \cite{breiman2001random}, per predire la prognosi di un paziente al carcinoma mammario invasivo usando dati provenienti dal gene BRCA.
	I dati a disposizione sono stati presi dal \textit{Cancer Genome Atlas} \cite{TCGASource}, programma scientifico di riferimento per la genomica del cancro. I dati si riferiscono alla versione del genoma umano hg19, versione pubblicata nel 2009 che contiene informazioni su sequenze di DNA umano, sequenze di geni, siti di regolazione genica e regioni non codificanti di DNA. I  dati sono composti da 4 \textit{dataset} distinti e comprendono: valori di espressione di proteine (proteins), RNA messaggero (mRNA), microRNA (miRNA) e le varianti nel numero di copie genetiche di DNA (CNV). 
	La presenza o assenza di un evento tumorale nel paziente sono invece state ottenute grazie all'utilizzo di un \textit{dataset} curato manualmente, noto come TCGA-CDR \cite{LIU2018400}.
	
	In particolare si è voluto verificare se l'utilizzo di dati multi-omici \cite{Hasin2017} riescano ad aumentare la precisione nella previsione della prognosi. Questo tipo di dati sono un insieme che contiene le variazioni molecolari su più livelli quali: genomica, epigenomica, trascrittomica, proteomica, metabolomica e microbiotica. Dopo una prima fase di \textit{preprocessing}, sono state utilizzate diverse tecniche di riduzione della dimensionalità sia prese singolarmente sia come concatenazione di più tecniche. Lo scopo è stato, in entrambe le soluzioni, cercare di semplificare il \textit{dataset} pur mantenendo contemporaneamente le informazioni più importanti e significative. Successivamente sono stati eseguiti gli esperimenti utilizzando le diverse configurazioni di tecniche di riduzione della dimensionalità applicate ai singoli \textit{dataset} e a ulteriori \textit{dataset} ottenuti mediante la concatenazione degli stessi al fine di cercare la miglior configurazione possibile.
	Per evitare di sovrastimare la capacità del modello di generalizzare su nuovi dati è stata usata una \textit{Stratified 10-fold cross-validation} in modo da non utilizzare lo stesso insieme di dati per addestrare e valutare il modello. Questa tecnica è stata anche utilizzata all'interno del \textit{tuning} degli iperparametri.
	Il \textit{dataset} fornito era sbilanciato così si è scelto di utilizzare il valore dell'area sotto la curva \textit{precision-recall} (AUPRC) sia come metrica per valutare la bontà di generalizzazione del modello sia per fare il \textit{tuning} degli iperparametri poiché è una metrica robusta per \textit{dataset} così composti.
	
	Per eseguire questo studio sono state utilizzate diverse tecnologie, prima tra tutte il linguaggio usato è stato \textit{Python} 3.10.6, unitamente a diversi pacchetti specifici per l'analisi dei dati quali: \textit{pandas} e \textit{numpy}. Pacchetti per applicare le tecniche di riduzione della dimensionalità quali: \textit{scipy}, \textit{umap}, \textit{minepy}, \textit{pymrmr} e \textit{boruta}. Infine \textit{sklearn} per l'applicazione specifica di tecniche di apprendimento automatico.
	Come ambiente di lavoro è stato usato \textit{Jupyter Notebook}, mentre come tool di \textit{versioning}, \textit{git} unitamente a \textit{github}. 
	 
	Utilizzando come metrica l'AUPRC, la maggior parte degli esperimenti effettuati ha avuto un risultato di previsione della malattia superiore rispetto a una diagnosi casuale, in particolare la seguente concatenazione di tecniche: Pearson + calcolo dimensionalità intrinseca + umap raccoglie i risultati migliori su tutti i \textit{dataset} forniti e creati mediante concatenazione. In particolare usando il \textit{dataset} miRNA è stato possibile classificare come sani tutti i pazienti effettivamente sani (550 su 550) e su $77$ pazienti malati il classificatore ne ha riconosciuti come tali $54$.
	Non c'è però evidenza concreta che l'utilizzo di dati multi-omici riesca ad aumentare la precisione della prognosi della malattia. Uno dei problemi maggiori è stata la composizione del \textit{dataset} che era sbilanciato tra pazienti sani e malati, anche se in campo medico questo è relativamente comune. Su $627$ osservazioni solo $77$ erano di pazienti effettivamente malati. Questa scarsità di osservazioni positive non ha di certo aiutato il modello a riuscire a generalizzare a dovere.
	
	
	\bibliography{bibliografia}
	\bibliographystyle{ieeetr}
	
	
\end{document}